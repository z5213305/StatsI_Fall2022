\documentclass[12pt,letterpaper]{article}
\usepackage{graphicx,textcomp}
\usepackage{natbib}
\usepackage{setspace}
\usepackage{fullpage}
\usepackage{color}
\usepackage[reqno]{amsmath}
\usepackage{amsthm}
\usepackage{fancyvrb}
\usepackage{amssymb,enumerate}
\usepackage[all]{xy}
\usepackage{endnotes}
\usepackage{lscape}
\newtheorem{com}{Comment}
\usepackage{float}
\usepackage{hyperref}
\newtheorem{lem} {Lemma}
\newtheorem{prop}{Proposition}
\newtheorem{thm}{Theorem}
\newtheorem{defn}{Definition}
\newtheorem{cor}{Corollary}
\newtheorem{obs}{Observation}
\usepackage[compact]{titlesec}
\usepackage{dcolumn}
\usepackage{tikz}
\usetikzlibrary{arrows}
\usepackage{multirow}
\usepackage{subcaption}
\usepackage{xcolor}
\newcolumntype{.}{D{.}{.}{-1}}
\newcolumntype{d}[1]{D{.}{.}{#1}}
\definecolor{light-gray}{gray}{0.65}
\usepackage{url}
\usepackage{listings}
\usepackage{color}

\definecolor{codegreen}{rgb}{0,0.6,0}
\definecolor{codegray}{rgb}{0.5,0.5,0.5}
\definecolor{codepurple}{rgb}{0.58,0,0.82}
\definecolor{backcolour}{rgb}{0.95,0.95,0.92}

\lstdefinestyle{mystyle}{
	backgroundcolor=\color{backcolour},   
	commentstyle=\color{codegreen},
	keywordstyle=\color{magenta},
	numberstyle=\tiny\color{codegray},
	stringstyle=\color{codepurple},
	basicstyle=\footnotesize,
	breakatwhitespace=false,         
	breaklines=true,                 
	captionpos=b,                    
	keepspaces=true,                 
	numbers=left,                    
	numbersep=5pt,                  
	showspaces=false,                
	showstringspaces=false,
	showtabs=false,                  
	tabsize=2
}
\lstset{style=mystyle}
\newcommand{\Sref}[1]{Section~\ref{#1}}
\newtheorem{hyp}{Hypothesis}

\title{Example PS Response}
\date{Jeffrey Ziegler}
\author{Applied Stats/Quant Methods 1}

\begin{document}
	\maketitle
	
	\section*{Instructions}

 \textit{This is a template from which you can reference to create your own responses in \texttt{R} and \texttt{Latex}.}

\vspace{1cm}
\section*{Question 1 }

\textit{Example Prompt.}\\

\vspace{.25cm}

Maybe I need to load in my dataset first, so let's show you how to 'present' that information. You can merely 'show' us that you read in the data by 'printing' your code. 

\lstinputlisting[language=R, firstline=40, lastline=40]{PS1_answersJZ.R}  

Notice, I'm reading in only one line of code from the answers in my .R file \texttt{\lstinputlisting[language=R, firstline=40, lastline=40]{PS1\_answersJZ.R} }).

\vspace{.5cm}

You could also copy your results using the \texttt{verbatim} environment like this":

\begin{verbatim}
STATE          Y                X1             X2              X3       
AK     : 1   Min.   : 49.00   Min.   :1053   Min.   :334.0   Min.   :326.0  
AL     : 1   1st Qu.: 68.25   1st Qu.:1698   1st Qu.:374.2   1st Qu.:426.2  
AR     : 1   Median : 81.00   Median :1897   Median :395.0   Median :568.0  
AZ     : 1   Mean   : 85.04   Mean   :1912   Mean   :404.7   Mean   :561.7  
CA     : 1   3rd Qu.:102.00   3rd Qu.:2096   3rd Qu.:419.5   3rd Qu.:661.2  
CO     : 1   Max.   :142.00   Max.   :2817   Max.   :637.0   Max.   :899.0  
\end{verbatim}
\vspace{.5cm}


\begin{figure}[h!]\centering

	\caption{\footnotesize Example from base plot in R.}
		\label{fig:plot_1}
	\includegraphics[width=.85\textwidth]{plot_example.pdf}
\end{figure}



\end{document}
